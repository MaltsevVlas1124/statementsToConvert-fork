\begin{problemAllDefault}{Логічний куб}

% % % \myflfigaw{\raisebox{-84pt}[0pt][72pt]{\begin{mfpic}[15]{0}{5}{0}{5}
% % % \polygon{(0,0),(3,0),(3,3),(0,3)}
% % % \lines{(3,0),(5,2),(5,5),(3,3)}
% % % \lines{(0,3),(2,5),(5,5),(3,3)}
% % % \dotted\lines{(2,2),(0,0)}
% % % \dotted\lines{(2,2),(2,5)}
% % % \dotted\lines{(2,2),(5,2)}
% % % \tlabel[tc](0,0){$\phantom{9\over a}$\font\mysize = cmmi10 scaled 1440 \mysize e$\phantom{9\over a}$}
% % % \tlabel[tc](2,2){$\phantom{9\over a}$\font\mysize = cmmi10 scaled 1440 \mysize f$\phantom{9\over a}$}
% % % \tlabel[tc](5,2){$\phantom{9\over a}$\font\mysize = cmmi10 scaled 1440 \mysize g$\phantom{9\over a}$}
% % % \tlabel[tc](3,0){$\phantom{99\over aa}$\font\mysize = cmmi10 scaled 1440 \mysize h$\phantom{9\over a}$}
% % % \tlabel[bc](0,3){$\phantom{9\over a}$\font\mysize = cmmi10 scaled 1440 \mysize a$\phantom{9\over a}$}
% % % \tlabel[bc](2,5){$\phantom{9\over a}$\font\mysize = cmmi10 scaled 1440 \mysize b$\phantom{9\over a}$}
% % % \tlabel[bc](5,5){$\phantom{9\over a}$\font\mysize = cmmi10 scaled 1440 \mysize c$\phantom{9\over a}$}
% % % \tlabel[bc](3,3){$\phantom{9\over a}$\font\mysize = cmmi10 scaled 1440 \mysize d$\phantom{9\over a}$}
% % % \end{mfpic}}}
% % % Логічний куб\nolinebreak[3] --- це куб, у\nolinebreak[3] вершинах якого знаходяться значення 0\nolinebreak[3] (\texttt{false}) або\nolinebreak[1] 1\nolinebreak[3] (\texttt{true}). Потрібно знайти шлях від однієї заданої вершини до\nolinebreak[2] іншої; якщо такого шляху не\nolinebreak[3] існує, то вивести відповідне повідомлення. В\nolinebreak[3] кубі можна проходити через усі р\'{е}бра, а\nolinebreak[3] також через вершини, значення яких рівне~1.

{
\def\logCubeFirstParagraph{Логічний куб\nolinebreak[3] --- це куб, у\nolinebreak[3] вершинах якого знаходяться значення 0\nolinebreak[3] (\texttt{false}) або\nolinebreak[1] 1\nolinebreak[3] (\texttt{true}). Потрібно знайти шлях від однієї заданої вершини до\nolinebreak[2] іншої; якщо такого шляху не\nolinebreak[3] існує, то вивести відповідне повідомлення. В\nolinebreak[3] кубі можна проходити через усі р\'{е}бра, а\nolinebreak[3] також через вершини, значення яких рівне~1.}
\def\logCubeSecondParagraph{Напишіть програму, яка знаходить шлях між двома вершинами (якщо він існує), та виводить його у вигляді послідовності вершин куба. Гарантовано, що ці дві задані вершини мають значення 1.}
\def\logCubeMfPic{\begin{mfpic}[15]{0}{5}{0}{5}
\polygon{(0,0),(3,0),(3,3),(0,3)}
\lines{(3,0),(5,2),(5,5),(3,3)}
\lines{(0,3),(2,5),(5,5),(3,3)}
\dotted\lines{(2,2),(0,0)}
\dotted\lines{(2,2),(2,5)}
\dotted\lines{(2,2),(5,2)}
\tlabel[tc](0,0){$\phantom{9\over a}$\font\mysize = cmmi10 scaled 1440 \mysize e$\phantom{9\over a}$}
\tlabel[tc](2,2){$\phantom{9\over a}$\font\mysize = cmmi10 scaled 1440 \mysize f$\phantom{9\over a}$}
\tlabel[tc](5,2){$\phantom{9\over a}$\font\mysize = cmmi10 scaled 1440 \mysize g$\phantom{9\over a}$}
\tlabel[tc](3,0){$\phantom{99\over aa}$\font\mysize = cmmi10 scaled 1440 \mysize h$\phantom{9\over a}$}
\tlabel[bc](0,3){$\phantom{9\over a}$\font\mysize = cmmi10 scaled 1440 \mysize a$\phantom{9\over a}$}
\tlabel[bc](2,5){$\phantom{9\over a}$\font\mysize = cmmi10 scaled 1440 \mysize b$\phantom{9\over a}$}
\tlabel[bc](5,5){$\phantom{9\over a}$\font\mysize = cmmi10 scaled 1440 \mysize c$\phantom{9\over a}$}
\tlabel[bc](3,3){$\phantom{9\over a}$\font\mysize = cmmi10 scaled 1440 \mysize d$\phantom{9\over a}$}
\end{mfpic}}

\ifAfour
\logCubeFirstParagraph
\par
\logCubeSecondParagraph
\myflfigaw{\hspace*{-1em}\logCubeMfPic}
\else
\mytextandpicture{\logCubeFirstParagraph}
{\raisebox{-84pt}[0pt][72pt]{\logCubeMfPic}}
\par
\logCubeSecondParagraph
\fi

}

\InputFile
В\nolinebreak[3] першому рядку задаються через пробіл дві вершини куба, це\nolinebreak[1] можуть бути дві з\nolinebreak[3] наступних маленьких латинських літер: $a$,\nolinebreak[3] $b$, $c$, $d$, $e$, $f$, $g$,\nolinebreak[3] $h$. В\nolinebreak[3] наступному рядку, послідовно записуються значення кожної з вершин куба (0\nolinebreak[2] або\nolinebreak[3] 1). Значення у\nolinebreak[2] вершинах перелічені в\nolinebreak[2] алфавітному порядку.

\OutputFile\phantomsection\label{text:log-cude-as-example-of-non-unique-correct-answer}
Якщо шлях існує, то\nolinebreak[2] вивести (без\nolinebreak[2] пробілів) послідовність\linebreak[1] маленьких латинських літер мінімальної довжини, які\nolinebreak[2] визначають шуканий\nolinebreak[2] шлях. Якщо\nolinebreak[2] такого шляху не\nolinebreak[3] існує, то\nolinebreak[2] вивести рядок\nolinebreak[3] ``\texttt{NO}''.

\Examples
\ifAfour\hspace{-1em}\fi
\begin{exampleSimple}{3em}{3em}%
\exmp{e d
10011011}{ead}%
\end{exampleSimple}
\ifAfour\hspace{-1em}\fi
\begin{exampleSimple}{3em}{3em}%
\exmp{e d
00011110}{NO}%
\end{exampleSimple}
\ifAfour\hspace*{-1em}\fi

\end{problemAllDefault}


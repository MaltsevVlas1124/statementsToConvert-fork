\Tutorial
На\nolinebreak[3] жаль, в умові була допущена помилка: кількість командирів (офіцерів) у одному місці називається~\verb"c", в\nolinebreak[3] інших~\verb"k".
На\nolinebreak[3] щастя, цю помилку було досить швидко помічено, причому її вдалося швидко нівелювати шляхом того, щоб вважати правильними і\nolinebreak[3] розв'язки, де\nolinebreak[3] ця\nolinebreak[3] змінна називається~\verb"k", і\nolinebreak[3] розв'язки, де\nolinebreak[3] вона називається~\verb"c". Це~саме стосується й наступної задачі.

З~останнього абзацу умови та здорового глузду легко побачити, що треба подивитися, скільки груп можна сформувати, беручи до уваги лише обмеження за кількістю командирів, скільки\nolinebreak[3] --- лише за кількістю рядових, і взяти мінімум. Тобто, виходить \texttt{min(k,\nolinebreak[3] r//2)}, бо\nolinebreak[2] всі\nolinebreak[2] \texttt{k}\nolinebreak[2] офіцерів цілком можуть одночасно взяти по групі, а\nolinebreak[2] з\nolinebreak[2] \texttt{r}\nolinebreak[2] рядових гарантовано можна сформувати \texttt{r//2} груп, і ніяк не~можна сформувати (одночасно!) ще~більше. 
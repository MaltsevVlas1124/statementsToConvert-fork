\begin{problemAllDefault}{День програміста}

\ifAfour
\myflfigaw{\ifnum\pdfoutput>0
\begin{minipage}{0.5\textwidth}
\vbox to 0.5\textwidth{\vspace*{-9pt}\noindent\includegraphics[width=\textwidth,keepaspectratio=true]{cal2017patched.pdf}\vss}
\end{minipage}
\else
\begin{minipage}{0.5\textwidth}
\begin{tiny}\colorbox{yellow}{Run not latex but pdflatex to insert picture}\end{tiny}
\begin{small}\colorbox{yellow}{Run not latex but pdflatex to insert picture}\end{small}
\ifnum\number\month > 7 \ERROR \fbox{Run not latex but pdflatex to insert picture}\fi
\end{minipage}
\fi}
\fi

День програміста припадає на \mbox{256-й} день року, у невисокосний рік це 13 вересня, а у високосний~\nolinebreak[3] --- 12. 

Аналогічно пропонується розпізнати число та номер місяця, що припадає на будь-який день за номером~\texttt{n} у не~високосному 2017~році.



\ifAfour\else
\ifnum\pdfoutput>0
\noindent\includegraphics[width=\textwidth,keepaspectratio=true]{cal2017patched.pdf}
\else
\begin{tiny}\colorbox{yellow}{Run not latex but pdflatex to insert picture}\par\end{tiny}
\begin{small}\colorbox{yellow}{Run not latex but pdflatex to insert picture}\end{small}
\ifnum\number\month > 7 \ERROR \fbox{Run not latex but pdflatex to insert picture}\fi
\fi
\fi
% % % \begin{color}{red}
% % % Переробити бо тут календар обрізаний знизу\TODO
% % % \end{color}    

\InputFile
Єдине число від~1 до~365\nolinebreak[3] --- номер дня у році.

\OutputFile
Два числа в один рядок через пропуск\nolinebreak[3] --- відповідні день та місяць.

\ifAfour\else
\vspace{-0.25\baselineskip}
\fi

\Example
\hspace{-1em}\begin{exampleSimple}{5em}{5em}
\exmp{256}{13 9}\end{exampleSimple}\hspace*{-1em}

\end{problemAllDefault}


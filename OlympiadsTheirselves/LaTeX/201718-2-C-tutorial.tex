\Tutorial
\MyParagraph{Скільки повинен набирати очевидний розв'язок <<перебрати всі ч\'{и}сла від 1 до $N{-}1$, розібрати кожне на цифри, й подивитися, чи всі ці цифри з набору 1,~2,~4,~8?>>} Приблизно 100 балів з 300. Точна величина залежить від мови програмування\linebreak[2] (це\nolinebreak[3] впливає на\nolinebreak[2] швидкість) та деталей реалізації цього підходу.

\MyParagraph{А як набрати більше?}
Ця\nolinebreak[3] задача має багато спільного з задачами <<Генератор паролів>> (стор.~\mbox{\pageref{text:201415-3-C-start}--\pageref{text:201415-3-C-finish}}) та <<Сучасне мистецтво>> (стор.~\mbox{\pageref{text:201516-2-C-start}--\pageref{text:201516-2-C-finish}}). Деталі додумайте самостійно, враховуючи це та наведений далі приклад ручного аналізу. (Можливі й інші розв'язки цієї задачі; деякі з них теж достатньо ефективні, щоб пройти всі тести, вклавшись у обмеження часу, але ж далеко\nolinebreak[3] не\nolinebreak[3] всі.)

\MyParagraph{Навіщо виділено випадок $N=10^{T}$?}
У цьому частковому випадку задача має простий комбінаторний розв'язок: $4\dib{{+}}{4^2\,{+}\,\dots}\dib{{+}}4^{T}$, або $4\,{\cdot}\,\frac{4^T-1}{4-1}$. Наприклад, при ${N\,{=}\,10000}$ можна сказати, що відповіддю є сумарна кількість \mbox{1-знач}\-них, \mbox{2-знач}\-них, \mbox{3-знач}\-них та \mbox{4-знач}\-них <<цікавих>> чисел. Однозначних <<цікавих>> чисел чотири, бо це і є цифри 1,~2,\nolinebreak[2] 4,~8. Двозначних <<цікавих>> чисел шіст\-над\-цять; це можна підтвердити їх повним переліком
(11,\nolinebreak[3]
12,
14,
18,
21,
22,
24,
28,
41,
42,
44,
48,
81,
82,
84,\nolinebreak[3]
88),
але % набагато 
корисніше отримати цей результат так: є дві позиції, на кожну можна поставити будь-яку з чотирьох цифр 1, або~2, або~4, або~8; кожна з цифр на першій позиції може поєднуватися з будь-якою з цифр на другій позиції\nolinebreak[3] --- отже, % загальна 
кількість пар може бути обчислена як ${4\times4}\dib{{=}}16$. Аналогічно, \mbox{3-знач}\-них <<цікавих>> чисел є ${4^3\,{=}\,64}$, \mbox{4-знач}\-них ${4^4\,{=}\,256}$, а менших за $10^4$ є $4\dib{{+}}16\dib{{+}}64\dib{{+}}256\dibbb{{=}}340$.

\MyParagraph{Приклад ручного аналізу цієї задачі.}
Знайдемо кількість <<цікавих>> чисел, менших~8432.
Це,\nolinebreak[3] зокрема, всі 1-, 2- та \mbox{3-значні} <<цікаві>>, їх ${4\,{+}\,4^2\,{+}\,4^3}\dib{{=}}84$.
Крім того, раз старша цифра дорівнює~8, треба врахувати ще всі \mbox{4-значні} <<цікаві>>, що починаються з 1,~2 або~4 (їх\nolinebreak[3] у\nolinebreak[2] кожній з цих трьох груп по $4^3{=}64$, тобто всього~192; разом з раніше згаданими це $84\dib{{+}}192\dib{{=}}276$), і додати до них ті \mbox{4-значні} <<цікаві>>, які починаються з~8 і~при цьому менші~8432. Оскільки початкова~<<8>> тепер зафіксована, вона не~впливає на кількість, можна рахувати все\nolinebreak[2] одно як кількість усіх \mbox{3-нач}\-них <<цікавих>>, менших~432. Цілком аналогічно виділяємо окремо всі \mbox{3-значні} <<цікаві>>, що починаються з 1 або~2 (їх\nolinebreak[3] у\nolinebreak[2] кожній з цих двох груп по\nolinebreak[2] ${4^2\,{=}\,16}$, тобто всього~32; разом з раніше згаданими це $276\dib{{+}}32\dib{{=}}308$). Лишається порахувати кількість \mbox{3-знач}\-них <<цікавих>>, які починаються з~4 і~при цьому менші~432. Знову, початкова~<<4>> зафіксована й не~впливає на кількість, так що рахуємо кількість 2-значних <<цікавих>>, менших~32. Цілком аналогічно виділяємо окремо всі \mbox{2-значні} <<цікаві>>, що починаються з 1 або~2 (їх\nolinebreak[3] у\nolinebreak[2] кожній з цих двох груп по ${4^1\,{=}\,4}$, тобто всього~8; разом з раніше згаданими це $308\dib{{+}}8\dib{{=}}316$). Тільки тепер, хоча цифри ще~не~скінчилися, процес обривається, бо <<цікаві>> числа все\nolinebreak[3] одно не~можуть містити цифру~3. Так\nolinebreak[3] що відповіддю є~316.

\MyParagraph{А чим відрізняється останній блок від передостаннього?}
З\nolinebreak[3] точки зору пропонованого алгоритму\nolinebreak[3] --- майже нічим. Але відмінність є з точки зору іншого алгоритму: генерувати всі підряд лише <<цікаві>> ч\'{и}сла, доки не~дійде до вказаної меж\'{і}. Такий алгоритм може пройти тести передостаннього блоку, але тести останнього повинні не~вкладатися в ліміт часу.
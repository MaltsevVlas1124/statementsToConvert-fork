{

\PrintEjudgeConstraintsfalse

\begin{problemAllDefault}{Патрульні групи--1}
\label{problem:201718-oioi-A-patrol-group-1}

Патрульна група повинна складатися з одного командира та рівно двох рядових. Загальна кількість офіцерів, які можуть виконувати обов'язки командира патрульної групи, становить~\texttt{c}. Загальна кількість рядових, з яких можуть вибиратися підлеглі патрульні, становить~\texttt{r}. Офіцер не~може виконувати обов'язки рядового, а~рядовий не~може виконувати обов'язки командира групи.

Напишіть \emph{вираз}, котрий знаходитиме, яку максимальну \emph{кількість патрульних груп} можна сформувати \emph{одночасно}.

Кількість офіцерів, які можуть виконувати обов'язки командира патрульної групи, повинна позначатися обов'язково змінною~\texttt{k}.
Загальна кількість рядових, з яких можна вибирати підлеглих патрульних, повинна позначатися обов'язково змінною~\texttt{r}.
Обидві кількості гарантовано натуральні (цілі додатні).

У~цій задачі треба здати не~програму, а~вираз: 
вписати його (сам вираз, не~назву файлу) у~відповідне поле перевіряючої системи
і відправити на~перевірку. Правила запису виразу:
можна використовувати десяткові ч\'{и}сла, арифметичні дії ``\verb"+"''~(плюс), 
``\verb"-"''~(мінус), ``\verb"*"''~(множення), ``\verb"/"''~(ділення дробове, наприклад, \verb"17/5"${=}3{,}4$), ``\verb"//"''~(ділення цілочисельне, наприклад, \verb"17//5"${=}3$), круглі дужки ``\verb"("''\nolinebreak[2] та~``\verb")"'' 
для групування та зміни порядку дій, а~також функції \verb"min" та \verb"max".
Формат запису функцій: після \verb"min" або \verb"max" відкривна кругла дужка, потім перелік (через кому) виразів, від яких береться мінімум або максимум, потім закривна кругла дужка.
 Дозволяються пропуски (пробіли), 
але, звісно, не~всер\'{е}дині чисел і не~всер\'{е}дині імен \verb"min" та \verb"max".

Наприклад: можна здати вираз 
\verb"r/k"
і отримати 4~бали зі~100, бо він неправильний, 
але все~ж іноді відповідь випадково збігається з~правильною. 
А~<<цілком аналогічний>> вираз 
\verb"b/a"
буде оцінений на~0~балів, 
бо~змінні повинні називатися \verb"k" та~\verb"r", а~не~\verb"a"~та~\verb"b".

Для кращого розуміння умови задачі та суті її відмінності від наступної, наведемо також приклад: при 
$\texttt{k}{=}3$, 
$\texttt{r}{=}3$
числова відповідь дорівнює~1, бо хоч і є аж три можливих командири, але трьох рядових не~вистачає на формування навіть двох груп по~двоє рядових у кожній.

\end{problemAllDefault}

}
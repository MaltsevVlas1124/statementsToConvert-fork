\begin{problem}{Двійкова система числення}{binary.in}{binary.out}{1 сек}{64 Мб}

Василько нещодавно вивчив двійкову систему числення на уроці інформатики. Розв'язуючи
своє завдання, він помітив цікаву особливість двійкових чисел, якщо додати одиницю до
якогось числа, то в результаті отримаємо число, яке відрізняється від попереднього тим, що
декілька його молодших розрядів містять обернене значення. Наприклад, додавши до числа
$01\underline{0}_2$ одиницю, отримаємо число $01\underline{1}_2$, яке відрізняється лише одним молодшим розрядом. А\nolinebreak[3]
додавши до числа $0\underline{011}_2$ одиницю, отримаємо число $0\underline{100}_2$, яке вже відрізняється трьома
молодшими розрядами.

Василько вирішив написати програму, яка починаючи з якогось числа~$A$, буде додавати до
нього по одиниці, поки не утвориться число~$B$. При цьому програма повинна підрахувати
суму кількостей відмінних розрядів після кожної операції додавання.

Але тут пролунав дзвінок і Василько так і не встиг написати програму.

\Task 
Напишіть програму \texttt{binary}, яку Василько так і не встиг написати.

\InputFile
Перший рядок файлу \texttt{binary.in} містить два натуральних числа $A$ та $B$ ($1\dib{{\<}}A\dib{{<}}B\dib{{\<}}10^{10}$). Кожне число записане в десятковій системі числення.

\OutputFile
Ваша програма повинна створити текстовий файл \texttt{binary.out} і вивести туди
єдине число, яке є відповіддю на задачу, теж в десятковій системі числення.

\Example
\begin{exampleWidthsAndDefaultFileNames}{2em}{2em}%
\exmp{2 4}{4}%
\end{exampleWidthsAndDefaultFileNames}

\end{problem}


\ifAfour
\myflfigaw{\hspace*{-3mm}\input 201617-2-A-statement-picture}
\fi

{

\PrintEjudgeConstraintsfalse

\begin{problemAllDefault}{Квартали}

Місто з квадратними кварталами має вигляд прямокутника, розмір якого зі сходу на захід $n$~кварталів, а\nolinebreak[3] з\nolinebreak[3] півночі на південь\nolinebreak[3] --- $m$~кварталів. Як\nolinebreak[2] між\nolinebreak[2] кварталами, так\nolinebreak[3] і по\nolinebreak[3] краям міста прокладені вулиці ширини~$a$. Самі квартали (без вулиць) мають розмір\nolinebreak[3] $b\*b$.

Напишіть \emph{вираз}, залежний від \texttt{a},~\texttt{b},\nolinebreak[2] \texttt{n},~\texttt{m}, за\nolinebreak[3] яким можна обчислити сумарну площу всіх доріг. Правила запису виразу спільні для більшості мов програмування: дозволені десяткові цілі числа, операції \verb"+" (плюс), \verb"–" (мінус), \verb"*" (помножити), круглі дужки для задання порядку дій; множення треба писати явно (тобто, не~можна писати добуток як \verb"mn", лише\nolinebreak[2] як\nolinebreak[3] \verb"m*n").

Здати треба вираз, а~не~програму. В~ejudge треба вписати у відповідне поле сам\nolinebreak[2] вираз, не~створюючи ніякого файлу-розв’язку. Змінні \texttt{a},~\texttt{b},\nolinebreak[2] \texttt{n},~\texttt{m} з умови задачі повинні називатися саме \texttt{a},~\texttt{b},\nolinebreak[2] \texttt{n},~\texttt{m}, перейменовувати не~можна.

\ifAfour\else
\input 201617-2-A-statement-picture
\fi

Приклад: на рисунку зображено місто, де\nolinebreak[2] ${a\,{=}\,2}$, ${b\,{=}\,8}$, ${n\,{=}\,4}$, ${m\,{=}\,3}$; для такого міста, шукана загальна площа доріг дорівнює~576.

\end{problemAllDefault}

}



\myflfigaw{\mbox{\raisebox{\ifAfour -24pt \else -30pt\fi}[24pt][0pt]{\begin{mfpic}[6]{0}{10}{-3}{3}
\coords
\rotate{-15}
\circle{(3,0),3}
\circle{(8,0),2}
\circle{(5,0),5}
\endcoords
\end{mfpic}}}}% TODO: потребУє ретельного перегляду куди сАме вверсталося
\Tutorial
Два менші диски, яким заборонено накладатися, займають найменше місця, коли торкаються. 
Тому гранична ситуація, коли при хоч трохи меншому найбільшому диску вони вже звисають, а\nolinebreak[3] при рівно такому або більшому все гаразд, зображена на\nolinebreak[2] рисунку. Наприклад, \mbox{2-й} і \mbox{3-й} круги можна покласти поверх \mbox{1-го} тоді й тільки тоді, коли $R_1\dib{{\>}}R_2\dib{{+}}R_3$. Решта випадків аналогічні.
Щоб правильно виводити\nolinebreak[3] ``\texttt{NO}'', легше не~формулювати умову цього випадку, а\nolinebreak[3] зробити розгалуження вкладеними, щоб виводити\nolinebreak[3] ``\texttt{NO}'', коли не~виконалася жодна з трьох інших умов.
Реалізацію див.\nolinebreak[2] \IdeOne{rx2Odn}.
%%% Приклад програми-розв'язку\nolinebreak[3] --- \IdeOne{rx2Odn}

Можливий інший підхід\nolinebreak[3] --- спочатку знайти, який з\nolinebreak[3] дисків максимальний, а\nolinebreak[3] вже потім провести одне порівняння. Для даної задачі це погана ідея:\linebreak[1] і\nolinebreak[3] тому, що треба виводити номер максимального диску (ще\nolinebreak[3] й у\nolinebreak[3] вигляді ``1st''/``2nd''/``3rd''), і\nolinebreak[3] тому, що з'ясування, який диск максимальний, займе чи\nolinebreak[3] не\nolinebreak[3] більше дій, ніж розгляд випадків у~попередньому розв'язку. Але для багатьох інших задач \mbox{деякий} аналог др\'{у}гого підходу буває набагато кращим за аналіз випадків, аналогічний першому підходу.

%%% \makeTableLongtrue

\makeTableLongfalse

\begin{problemAllDefault}{Checker для ``Ракети''}

У~цій задачі Вам пропонується зробити те, що зазвичай роблять автори задач, а~не~учасники.
В~минулій задачі були обіцяні різні бали залежно від того, з якою точністю знайдено відповідь.
У~цій задачі, Вашій програмі будуть надані правильна відповідь попередньої задачі та відповідь учасника, а~Ваша програма повинна буде видати, як слід це оцінити.

\InputFile
Перший рядок вхідних даних являє собою правильну відповідь на задачу <<Ракета>>, з рівно 25 десятковими цифрами після крапки; гарантовано, що це буде значення відповіді, правильне (в~рамках наведених знаків, отже, з~абсолютною похибкою до~$10^{-25}$) для якихось вхідних даних попередньої задачі (яких с\'{а}ме вхідних даних, невідомо, але в рамках вказаних там обмежень). Надалі до значення з \mbox{1-го} рядка слід ставитися як до точної відповіді, ігноруючи, що воно сам\'{е} може бути заокругленням.

Др\'{у}гий рядок вхідних даних являє собою відповідь учасника. Він може містити абсолютно що завгодно, але гарантовано являє собою лише один рядок. Вхідні дані в~цілому гарантовано містять рівно два рядки, не~більше й не~менше. Кожен з рядків завершується символом переведення рядка (в~Unix-форматі).

\myflfigaw{\hspace*{-1em}\begin{tabular}{@{}c@{}}\Examples\\
\begin{exampleSimple}{16.25em}{5em}
\exmp{1.0000000000000000000000000\\
+1}{OK 10}
\exmp{1.0000000000000000000000000\\
1.0000000000000000000000000}{OK 10}
\exmp{1.0000000000000000000000000\\
1.0000000000000000000000000000000}{PE}
\exmp{1.0000000000000000000000000\\
1.0000000002718281828}{PT 6}
\exmp{0.0001000000000000000000000\\
0.0001000002718281828}{PT 3}
\exmp{0.0000120762762995368423496\\
+1.207627629953684235E-0005}{OK 10}
\exmp{0.0001000000000000000000000\\
\mbox{~~~}0.1000000000000000082e-3~}{OK 10}
\exmp{0.0001000000000000000000000\\
Маша їла кашу.}{PE}
\exmp{0.0001000000000000000000000\\
+++0.0001}{PE}
\exmp{0.0001000000000000000000000\\
Відповідь дорівнює: 0.0001}{PE}%
\end{exampleSimple}
\end{tabular}\hspace*{-1em}}

{\hyphenpenalty=-1
\OutputFile
Якщо довжина \mbox{2-го} рядка перевищує~30~символів (31~і~більше), виведіть~``\texttt{PE}''. Якщо \mbox{2-й} рядок не~є записом числ\'{а} (зокрема, якщо містить і запис числ\'{а}, і~ще~щось), теж виведіть~``\texttt{PE}''. 
Але якщо зайвими є лише пробіли (ASCII-код~32) перед числом та/або після числ\'{а}, вони повинні не~впливати на~результат\nolinebreak[3] --- зокрема, том\'{у}, що деякі версії Паскаля ставлять їх самі, як-то \mbox{``\texttt{~1.0000000000000000E-005}''} (без~лапок, але з пробілом).

Якщо \mbox{2-й} рядок є записом числ\'{а}, відповідь Вашої програми повинна залежати від похибки. Позначимо значення числ\'{а} з \mbox{1-го} рядка (правильної відповіді) як~$a$, числ\'{а} з \mbox{2-го} рядка (відповіді учасника) як~$b$. При $\frac{|b-a|}{a}\,{<}\,10^{-15}$, слід вивести\nolinebreak[3] ``\texttt{OK~10}''. 
При $10^{-15}\,{<}\,\frac{|b-a|}{a}\,{<}\,10^{-9}$, слід вивести\nolinebreak[3] ``\texttt{PT~6}''. 
При $10^{-9}\,{<}\,\frac{|b-a|}{a}\,{<}\,10^{-3}$, слід вивести\nolinebreak[3] ``\texttt{PT~3}''. 
При $\frac{|b-a|}{a}\,{>}\,10^{-3}$, слід вивести~``\texttt{WA}''. (Тести \emph{не~містять} перевірок, що слід виводити, коли відносна похибка \emph{в~точності рівна} $10^{-15}$ (чи~$10^{-9}$, чи~$10^{-3}$). Головним чином тому, що при обчисленні значення похибки теж можливі похибки, й точної рівності все~одно не~виходить.)

Будь-яку з відповідей слід виводити в першому і єдиному рядку. Якщо відповідь містить і двобуквенний код, і число, вони повинні бути розділені одинарним пробілом. Самі двобуквенні коди повинні бути виведені великими латинськими буквами. 
``\texttt{PE}''~є\nolinebreak[3] скороченням від ``\underline{\textbf{P}}resentation \underline{\textbf{E}}rror'', 
``\texttt{PT}''\nolinebreak[3] --- від ``\underline{\textbf{P}}ar\underline{\textbf{t}}ial solution'', 
``\texttt{WA}''\nolinebreak[3] --- від ``\underline{\textbf{W}}rong \underline{\textbf{A}}nswer''.

}

\Scoring
Перші 10~тестів є тестами з умови й не~оцінюються безпосередньо (але, для отримання повного балу за задачу, програма повинна пройти їх усі).
40$\,$\% (100) балів припадають на потестове оцінювання (кожен з 25 тестів приносить або\nolinebreak[3] 4~\mbox{бали}\nolinebreak[3] з~\mbox{4-х}, або~0, інші тести на це не~впливають). Решта 60$\,$\% (150) балів припадають на поблокове оцінювання: 10~блоків, кожен блок дає або\nolinebreak[3] 15~\mbox{балів}\nolinebreak[3] з~\mbox{15-ти} (якщо пройдено \emph{всі} його тести), або~0. У~цій задачі, детальніша інформація про розподіл тестів між блоками, а~також про відмінності та залежності між різними блоками не~підлягає розголошенню до кінця туру.


\end{problemAllDefault}


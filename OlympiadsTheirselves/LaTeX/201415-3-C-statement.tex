\begin{problemAllDefault}{Генератор паролів --- password}

\label{label:201415-3-C-start}

У генераторі паролів закладена схема, яка за певними правилами утворює паролі із комбінацій цифр та великих літер англійського алфавіту. Цифри та літери в паролі можуть розташовуватись лише за зростанням (для літер зростання визначається алфавітом). Паролі містять, як мінімум, один символ (літеру або цифру). Якщо пароль містить і цифри, і літери, то цифри завжди йдуть після літер. Повторення символів не~допускається.

Наприклад: \texttt{CH15} --- коректний пароль, а \texttt{OLYMPIAD} --- некоректний пароль (оскільки літери розташовані не~в~алфавітному порядку).

Усі паролі генеруються послідовно у вигляді впорядкованого списку. Якщо два паролі містять різну кількість символів, то першим іде пароль з меншою кількістю. Якщо декілька паролів мають однакову кількість символів, то вони генеруються в алфавітному порядку (при цьому літери вважаються меншими за цифри).

Початок впорядкованого списку паролів має наступний вигляд: \texttt{A},~\texttt{B},~\dots, \texttt{Z}, \texttt{0}, \texttt{1}, \texttt{2},~\dots, \texttt{9}, \texttt{AB}, \texttt{AC},~\dots, \texttt{A9},~\texttt{BC},~\dots 

\Task Напишіть програму \texttt{password}, яка визначатиме $n$-ий пароль у списку паролів, який утворить генератор.

\InputFile Число $n$ ($1\<n\<10^{10}$).

\OutputFile Ваша програма має вивести $n$-ий пароль.

\Examples
\begin{exampleSimple}{2em}{2em}%
\exmp{1}{A}%
\exmp{37}{AB}%
\end{exampleSimple}

\Scoring Приблизно 50\% балів припадає на тести, в яких $n\dib{{\<}}50000$. Ще приблизно\nolinebreak[2] 25\%\nolinebreak[3] --- на тести, в яких $n\<10^7$. Решта (приблизно\nolinebreak[2] 25\%)\nolinebreak[3] --- на тести, в яких $10^9 \< n \< 10^{10}$.


\Note Англійський алфавіт має вигляд A~B~C D E F G H I J K L M N O P Q R S T U V W X~Y~Z. Літери алфавіту у таблиці ASCII йдуть поспіль, під номерами від 65~(``A'') до\nolinebreak[3] 90~(``Z''). Зростаюча послідовність цифр 0~1~2 3 4 5 6 7~8~9 також неперервна у таблиці ASCII (але\nolinebreak[3] не\nolinebreak[2] поспіль з\nolinebreak[3] літерами). Цифри йдуть під номерами від 48~(``0'') до\nolinebreak[3] 57~(``9'').


\end{problemAllDefault}

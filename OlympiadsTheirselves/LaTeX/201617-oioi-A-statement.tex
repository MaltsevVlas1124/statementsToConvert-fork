{

\PrintEjudgeConstraintsfalse

\begin{problemAllDefault}{Квартира}

Розглянемо \mbox{9-по}\-вер\-ховий житловий будинок, в~якому кілька під'їздів, 
і кожен поверх кожного під'їзду містить рівно~4~квартири. Таким чином, 
\mbox{1-й}\nolinebreak[3] поверх \mbox{1-го} під'їзду містить квартири \textnumero\textnumero$\,$1,~2,\nolinebreak[1] 3,~4;\hspace{0.5em plus 0.5em}
\mbox{2-й}\nolinebreak[3] поверх \mbox{1-го} під'їзду\nolinebreak[3] --- квартири \textnumero\textnumero$\,$5,~6,\nolinebreak[1] 7,~8;\hspace{0.25em plus 0.25em}
\dots;\hspace{0.5em plus 0.5em}
\mbox{9-й}\nolinebreak[3] поверх \mbox{1-го} під'їзду\nolinebreak[3] --- квартири \textnumero\textnumero$\,$33,~34,\nolinebreak[1] 35,~36;\hspace{0.5em plus 0.5em}
\mbox{1-й}\nolinebreak[3] поверх \mbox{2-го} під'їзду\nolinebreak[3] --- квартири \textnumero\textnumero$\,$37,~38,\nolinebreak[1] 39,~40,
і~так~далі.
%%%Загальна кількість під'їздів не~перевищує~20.

Напишіть \emph{вираз}, котрий за номером під'їзду, номером поверху та <<локальним>> номером квартири у~межах сходової клітки обчислюватиме <<глобальний>> номер квартири у~будинку.
%
Номер під'їзду повинен бути позначений обов'язково змінною \texttt{pid}, його значення гарантовано будуть від~1 до~20.
Поверх повинен бути позначений обов'язково змінною \texttt{pov}, його значення гарантовано будуть від~1 до~9.
Номер квартири у~межах сходової клітки повинен бути позначений обов'язково змінною \texttt{kv}, його значення гарантовано будуть від~1 до~4.

Ще раз: у цій задачі (єдиній з переліку) треба здати не~програму, а~вираз: 
вписати його (сам вираз, не~назву файлу) у~відповідне поле перевіряючої системи
і відправити на~перевірку. Правила запису виразу\nolinebreak[3] --- 
спільні для більшості мов програмування:
можна використовувати десяткові ч\'{и}сла, арифметичні дії ``\verb"+"''~(плюс), 
``\verb"-"''~(мінус), ``\verb"*"''~(множення), круглі дужки ``\verb"("'' та ``\verb")"'' 
для групування та зміни порядку дій. Дозволяються пропуски (пробіли), 
але, звісно, не~всер\'{е}дині чисел і не~всер\'{е}дині імен змінних.

Наприклад: можна здати вираз \verb"pid * pov * kv" і отримати 26~балів з~200, бо він неправильний, 
але все~ж іноді відповідь збігається з~правильною. 
А~<<цілком аналогічний>> вираз \verb"a*b*c" буде оцінений на~0~балів, 
бо~змінні повинні називатися  \verb"pid", \verb"pov", \verb"kv", а~не~\verb"a",~\verb"b",~\verb"c".

\end{problemAllDefault}

}
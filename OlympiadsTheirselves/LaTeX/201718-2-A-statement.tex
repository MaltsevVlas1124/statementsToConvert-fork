{

\PrintEjudgeConstraintsfalse

\begin{problemAllDefault}{Фарбування паралелепіпеда}

Є\nolinebreak[3] прямокутний паралелепіпед розмірами ${x\textnormal{~см}}\dib{{\times}}{y\textnormal{~см}}\dib{{\times}}{z\textnormal{~см}}$. 
Є\nolinebreak[3] три фарби, які мають різну ціну: найдешевша\nolinebreak[3] --- ціну $a$~коп/см${}^2$, середня\nolinebreak[3] --- ціну $b$~коп/см${}^2$, найдорожча\nolinebreak[3] --- ціну $c$~коп/см${}^2$.
%
Зверніть увагу, що ${a<b<c}$, але ніякого аналогічного співвідношення між $x$,~$y$,~$z$ не~гарантовано, вони можуть бути будь-якими.

Напишіть \emph{вираз}, залежний від \texttt{a},~\texttt{b},\nolinebreak[2] \texttt{с},\nolinebreak[2] \texttt{x},\nolinebreak[2] \texttt{y},~\texttt{z}, за\nolinebreak[3] яким можна обчислити \underline{\emph{мінімальну}} можливу вартість (у~копійках) фарбування паралелепіпеда, щоб протилежні грані були пофарбовані в\nolinebreak[3] однакові кольори, а\nolinebreak[3] не\nolinebreak[3] протилежні\nolinebreak[3] --- у\nolinebreak[3] різні.
(Гранями паралелепіпеда є прямокутники, їх шість, і їх можна розбити на три пари протилежних: наприклад, якщо покласти паралелепіпед на рівну поверхню, можна говорити про протилежну пару <<верхня та нижня грані>>, протилежну пару <<ліва та права грані>>, і протилежну пару <<передня та задня грані>>.)

Здати треба вираз, а~не~програму. В~ejudge треба вписати у відповідне поле сам\nolinebreak[2] вираз, не~створюючи ніякого файлу-розв’язку. Змінні \texttt{a},~\texttt{b},\nolinebreak[2] \texttt{с},\nolinebreak[2] \texttt{x},\nolinebreak[2] \texttt{y},~\texttt{z} з умови задачі повинні називатися саме \texttt{a},~\texttt{b},\nolinebreak[2] \texttt{с},\nolinebreak[2] \texttt{x},\nolinebreak[2] \texttt{y},~\texttt{z}, перейменовувати не~можна.
%
У~виразі можна використовувати десяткові ч\'{и}сла, дії ``\verb"+"''~(плюс), 
``\verb"-"''~(мінус), ``\verb"*"''~(множення), круглі дужки ``\verb"("''\nolinebreak[2] та~``\verb")"'' 
для групування та зміни порядку дій, а~також функції \verb"min" та \verb"max".
Формат запису функцій: після \verb"min" або \verb"max" відкривна кругла дужка, потім перелік (через кому) виразів, від яких береться мінімум чи максимум, потім закривна кругла дужка.
 Пропуски (пробіли) дозволяються, 
але\nolinebreak[2] не~всер\'{е}дині чисел і не~всер\'{е}дині імен \verb"min"\nolinebreak[2] та~\verb"max".

Наприклад, можна здати вираз \texttt{(a+b+c)*(min(x,y)+max(x,y,z))}; він непра\-виль\-ний за~смислом, але змінні названі правильно, функції використані дозволеним способом, і він іноді випадково дає правильний результат; цей розв'язок оцінюється на 11~балів з 200 можливих.

Наприклад, при 
${a\,{=}\,2}$, 
${b\,{=}\,3}$, 
${c\,{=}\,4}$, 
${x\,{=}\,y\,{=}\,z\,{=}\,1}$ 
числовий результат має дорівнювати~18 (на дві протилежні грані розмірами $1{\times}1$ йде фарба ціною~2, на ще дві протилежні, теж $1{\times}1$, фарба ціною~3, на ще дві протилежні, теж $1{\times}1$, ціною~4; всього 
$2{\times}2{\times}1{\times}1\dibbb{{+}}
2{\times}3{\times}1{\times}1\dibbb{{+}}
2{\times}4{\times}1{\times}1\dibbb{{=}}
{4\,{+}\,6\,{+}\,8}\dibbb{{=}}18$).
А,~наприклад, при 
${a\,{=}\,2}$, 
${b\,{=}\,3}$, 
${c\,{=}\,4}$, 
${x\,{=}\,20}$, 
${y\,{=}\,30}$, 
${z\,{=}\,40}$ 
числовий результат має дорівнювати~14400.

\end{problemAllDefault}

}
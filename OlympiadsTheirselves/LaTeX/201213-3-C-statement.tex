\begin{problem}{Декадні числа --- dnumbers}{dnumbers.dat}{dnumbers.sol}{3 сек}{64 Мб}

Декадні числа\nolinebreak[3] --- це цілі додатні числа в яких сума \mbox{$і$-тої} цифри зліва та \mbox{$і$-тої} цифри зправа завжди дорівнює~10. Наприклад число 13579 є декадним, так як 
${1\,{+}\,9}\dibbb{{=}}10$,\hspace{0.5em plus 0.5em} 
${3\,{+}\,7}\dibbb{{=}}10$,\hspace{0.5em plus 0.5em} 
${5\,{+}\,5}\dibbb{{=}}10$
(в~даному випадку цифра\nolinebreak[3] 5 є третьою зліва і одночасно третьою справа).

Перші кілька декадних чисел в порядку зростання:
5,\nolinebreak[3] 19,\nolinebreak[2] 28, 37, 46, 55, 64, 73, 82, 91,\nolinebreak[2] 159,~\dots

\Task
Напишіть програму \texttt{dnumbers}, яка б знаходила n-e в порядку зростання декадне число.

\InputFile
В єдиному рядку файлу \texttt{dnumbers.dat} міститься число $n$ ($1\dib{{\<}}n\dib{{\<}}2^{31}$).

\OutputFile
Ваша програма має створити текстовий файл \texttt{dnumbers.sol} і вивести туди єдине число, яке є шуканим \mbox{$n$-им} декадним числом.

% % % \vspace{-0.25\baselineskip}

\Examples
\hspace{-1em}
\begin{exampleWidthsAndDefaultFileNames}{2em}{2em}%
\exmp{2}{19}%
\end{exampleWidthsAndDefaultFileNames}
\hspace{-1.75em}
\begin{exampleWidthsAndDefaultFileNames}{2em}{2em}%
\exmp{11}{159}%
\end{exampleWidthsAndDefaultFileNames}
\hspace*{-1em}

% % % \vspace{-0.25\baselineskip}

\end{problem}

\Tutorial	Задача на реалізацію, тобто не~треба нічого придумувати, лише прочитати й реалізувати. Приблизно як стандартний підрахунок суми, тільки врахувати ще початкову кількість шкіл. Наприклад, див. \IdeOne{XbSTUE}\nolinebreak\hspace{0.125em}.
Про смисл \texttt{ifdef} та \texttt{assign} у цьому коді див.\nolinebreak[2] стор.~\pageref{text:FAQ-ifdef-short-example}. 

Цей\nolinebreak[2] код реалізований так, щоб рахувати суму <<на\nolinebreak[2] ходу>>, \emph{не}~зберігаючи всі вхідні дані у масиві. Але, при обмеженні ${K\,{\<}\,100}$ та автоматичній перевірці, учасник має повне право вибрати простіший для себе варіант (з~масивом чи\nolinebreak[3] без). На\nolinebreak[3] олімпіадах часто вимагають ефективний код, але це мають бути ситуації, де ефективний код легко розрізняється від неефективного. Ця\nolinebreak[3] ситуація такою\nolinebreak[2] не~є, бо нереально помітити зайву сотню \mbox{4-}\nolinebreak[3]байто\-вих чисел на~фоні виконуваного файлу, який займає значно більше місця у пам'яті. Якби хотіли розрізнити засобами автоматичної перевірки, чи\nolinebreak[2] зумів учасник реалізувати таку програму без\nolinebreak[2] масива, давали\nolinebreak[3] б обмеження не\nolinebreak[3] ${K\,{\<}\,100}$, а десь так ${K\,{\<}\,10^6}$ (чи\nolinebreak[2] навіть ${K\,{\<}\,10^7}$), і при цьому жорстке обмеження пам'яті.

А\nolinebreak[3] от забезпечити, щоб усі ч\'{и}сла поміщалися (без\nolinebreak[3] переповнень) у\nolinebreak[3] тип, треба. Для більшості сучасних мов програмування <<стандартне>> ціле число вміщає в себе максимальне для цієї задачі ${101\,{\cdot}\,10^6}$, для деяких мов тут може бути проблема (зокрема, для Паскаля; див.\nolinebreak[2] також стор.~\pageref{text:overflow-example} та~\pageref{text:notes-about-delphi-mode}).
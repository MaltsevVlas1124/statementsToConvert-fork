\begin{problemAllDefault}{Знакозмінна сума}%

% Напишіть програму, яка знаходитиме значення знакозмінної суми
% ${1\,+\,2\,-\,3}\dibbb{{+}}{4\,+\,5\,-\,6\,+\,7\,+\,\ldots}$
Напишіть програму, що обчислюватиме знакозмінну суму
${1+2-3}\dib{{+}}{4+5-6+7+\ldots\pm{}n}$
(доданки/від'ємники від~1 до~$n$; кратні~3 "--- від'ємні).
% \vspace{-1.75\baselineskip}
% 
% $$
% 1 + 2 - 3 + 4 + 5 - 6 + 7 + \ldots\quad\quad\textnormal{(всього $n$ доданків\&від'ємників)}.
% $$
%
% \vspace{-0.5\baselineskip}

\InputFile
Єдине натуральне число~$n$. % Обмеження на значення $n$ див.~далі.

\OutputFile
Виведіть єдине ціле число\nolinebreak[3] --- значення виразу.

\myflfigaw{\begin{minipage}{13.5em}
\Examples\\
\begin{exampleSimple}{5em}{7em}
\exmp{1}{1}%
\exmp{2}{3}%
\exmp{3}{0}%
\exmp{4}{4}%
\exmp{5}{9}%
\exmp{6}{3}%
\exmp{7}{10}%
\exmp{8}{18}%
\exmp{2021}{682425}%
\exmp{20211212}{68082198594168}%
\end{exampleSimple}
\end{minipage}}


\Scoring
Потестове (проходження кожного тесту оцінюється окремо, інші тести на це не~впливають). Перші 10 тестів є тестами з умови й не~оцінюються (але\nolinebreak[3] перевіряються, а\nolinebreak[3] детальний протокол показується). 

60 балів припадає на тести, де $n$ 
% перебуває в 
у
межах $9\dib{{\<}}n\dib{{\<}}20$. 

Ще 60 "--- на тести, де $100\dib{{\<}}n\dib{{\<}}3000$.

Ще 20 "--- на тести, де $10^5\dib{{\<}}n\dib{{\<}}{10^6}$.

Ще 60 "--- на тести, де $10^9\dib{{\<}}n\dib{{\<}}{2{\cdot}10^9}$.

% Ще 20\% балів припадає на тести, в яких $100\dib{{\<}}n\dib{{\<}}300$.

% Ще 20\% балів припадає на тести, в яких $1000\dib{{\<}}n\dib{{\<}}3000$.

% % \def\cdotShortSpace{{\phantom{{\cdot}}}{\cdot}{\phantom{{\cdot}}}}

% Ще 10\% балів припадає на тести, в яких $10^5\dib{{\<}}n\dib{{\<}}{2{\cdot}10^5}$.

% Ще 30\% балів припадає на тести, в яких $10^9\dib{{\<}}n\dib{{\<}}{2{\cdot}10^9}$.

\Notes
Перетворити вираз, щоб \mbox{його} обчислення стало швидшим, можна.
% Вивести формулу й закодувати її мовою програмування, теж можна. Просто писати розв'язок цієї задачі <<в~лоб>>, ні\'{я}к не~намагаючись оптимізувати, теж можна (але це не~для всіх тестів вкладеться в обмеження часу).
%
Ви не~повинні й не~можете писати різні програми для різних обмежень (це\nolinebreak[3] стосується й усіх інших задач, де~не~сказано протилежне). 
Вам просто повідомляють, скільки за які тести передбачено балів.



\end{problemAllDefault}



{

\begin{problemAllDefault}{Василько та циркуль--2}

Василько знову взяв великого циркуля та зайшов до іншої порожньої кімнати, підлога якої теж являє собою прямокутник, вершини якого мають координати $(0;0)$, $(A; 0)$, $(A; B)$, $(0; B)$, і знову має намір поставити циркуль на підлозі цієї кімнати в точці з координатами $(x;y)$ (яка  теж розміщена всер\'{е}дині (внутри, inside) цієї кімнати, тобто ${0\,{<}\,x\,{<}\,A}$, ${0\,{<}\,y\,{<}\,B}$) та побудувати к\'{о}ла (окружности, circles), радіусами $r$, $2r$, $3r$,~\dots{} (радіус кожного наступного на~$r$ більший за радіус попереднього). Тільки тепер виявилося, що підлога цієї кімнати має $N$ круглих дірок:
з~центром у точці $(x_1;y_1)$ та радіусом~$r_1$, 
з~центром у точці $(x_2;y_2)$ та радіусом~$r_2$,\nolinebreak[3] \dots,
з~центром у точці $(x_N;y_N)$ та радіусом~$r_N$. 

Василько дуже боїться зламати свій циркуль, тому пропускатиме ті к\'{о}ла, які хоча~б частково потрапляють у хоча~б одну з дірок, або торкаються до хоча~б однієї дірки (у~смислі, звичайному для <<к\'{о}ла торкаються зовнішнім/внутрішнім чином>>). За рахунок цього, кількість намальованих кіл може виявитися меншою, чим у задачі~\ref{problem:2020oioi-circles-in-room}. Можлива навіть така сумна ситуація, що точка з координатами $(x;y)$, куди Василько хотів поставити циркуль, сам\'{а} потрапляє у деяку з дірок чи на межу д\'{і}рки; тоді Василько не~зможе намалювати жодного к\'{о}ла.

Скільки повних кіл може побудувати в цій кімнаті Василько?

\InputFile
Перший рядок містить два числ\'{а} $A$ та~$B$.
Др\'{у}гий рядок містить два числ\'{а} $x$ та~$y$.
Третій рядок містить єдине число~$r$.
Смисл чисел $A$, $B$, $x$, $y$, $r$ точнісінько такий самий, як у задачі~\ref{problem:2020oioi-circles-in-room}; всі ці ч\'{и}сла є натуральними (цілими додатними); обмеження на діапазони чисел див.\nolinebreak[2] у\nolinebreak[3] розділі <<Оцінювання>>.
Четвертий рядок містить єдине ціле невід'ємне число~$N$\nolinebreak[3] --- кількість круглих дірок у підлозі.
Подальші $N$ рядків, з \mbox{5-го} по \mbox{$(N\,{+}\,4)$-й}, містять рівно по три числа кожен: $x_i$, $y_i$ та $r_i$ відповідної д\'{і}рки в підлозі. Всі ці $x_i$, $y_i$, $r_i$ теж є натуральними (цілими додатними); обмеження на діапазони теж див.\nolinebreak[2] у\nolinebreak[3] розділі <<Оцінювання>>. Гарантовано, що всі дірк\'{и} повністю розміщені всер\'{е}дині кімнати, та що різні дірк\'{и} не~можуть ні~перетинатися, ні~дотикатися, ні~бути вкладеними одна\nolinebreak[2] в\nolinebreak[3] іншу.

\OutputFile
Виведіть єдине ціле невід'ємне число\nolinebreak[3] --- кількість кіл, які зможе намалювати Василько.

\savebox{\mypictbox}{\mbox{\raisebox{-120pt}{\begin{mfpic}[1.25]{-5}{125}{-5}{95}
\axes
\dashed\lines{( -5, 10),(125, 10)}
\dashed\lines{( -5, 20),(125, 20)}
\dashed\lines{( -5, 30),(125, 30)}
\dashed\lines{( -5, 40),(125, 40)}
\dashed\lines{( -5, 50),(125, 50)}
\dashed\lines{( -5, 60),(125, 60)}
\dashed\lines{( -5, 70),(125, 70)}
\dashed\lines{( -5, 80),(125, 80)}
\dashed\lines{( -5, 90),(125, 90)}
%
\dashed\lines{( 10, -5),( 10, 95)}
\dashed\lines{( 20, -5),( 20, 95)}
\dashed\lines{( 30, -5),( 30, 95)}
\dashed\lines{( 40, -5),( 40, 95)}
\dashed\lines{( 50, -5),( 50, 95)}
\dashed\lines{( 60, -5),( 60, 95)}
\dashed\lines{( 70, -5),( 70, 95)}
\dashed\lines{( 80, -5),( 80, 95)}
\dashed\lines{( 90, -5),( 90, 95)}
\dashed\lines{(100, -5),(100, 95)}
\dashed\lines{(110, -5),(110, 95)}
\dashed\lines{(120, -5),(120, 95)}
%
\hatch\circle{(60,30),5}
\hatch\circle{(90,20),15}
\circle{(60,30),5}
\circle{(90,20),15}
%
\pen{1.5pt}
\polygon{(0,0),(120,0),(120,90),(0,90)}
\point{(40,45)}
\circle{(40,45), 8}
\circle{(40,45),16}
\circle{(40,45),32}
%
\tlabel[tc](0,-1){${}_{0}$}
\tlabel[tc](40,-1){${}_{40}$}
\tlabel[tc](120,-1){${}_{120}$}
%
\tlabel[cr](-1,0){${}_{0}$}
\tlabel[cr](-1,45){${}_{45}$}
\tlabel[cr](-1,90){${}_{90}$}
\end{mfpic}}}}

\Example
\begin{exampleSimpleThree}{6em}{3em}{0.35\textwidth}{}
\exmp{120 90
40 45
8
2
60 30 5
90 20 15}{3}{\usebox{\mypictbox}}%
\end{exampleSimpleThree}

\Scoring
Тест 1 є тестом з умови і сам по собі не~приносить балів.

Тести 2--11 оцінюються кожен окремо і незалежно від того, чи~пройшли інші тести; можуть принести по 10~балів (2\%) кожен. 
У\nolinebreak[3] них виконуються такі додаткові обмеження: 
$1\dib{{<}}x\dib{{<}}A\dib{{<}}200$,\hspace{0.5em plus 0.5em}
$1\dib{{<}}y\dib{{<}}B\dib{{<}}200$,\hspace{0.5em plus 0.5em}
$0\dib{{\<}}N\dib{{\<}}20$.

Тести 12--15 оцінюються блоком (тут і далі це означає, що бали нараховуються, лише якщо \underline{\emph{всі}} тести цього діапазону пройшли успішно), завжди (незалежно від того, чи~пройшли попередні тести). 
У\nolinebreak[3] них виконуються такі додаткові обмеження: 
$1\dib{{<}}x\dib{{<}}A\dib{{<}}200$,\hspace{0.5em plus 0.5em}
$1\dib{{<}}y\dib{{<}}B\dib{{<}}200$,\hspace{0.5em plus 0.5em}
${N\,{=}\,0}$ (дірок нема).\linebreak[2]\hspace{0.5em plus 0.5em}
Цей\nolinebreak[3] блок може принести 25~балів (5\%).

Тести 16--20 оцінюються блоком, лише якщо успішно пройшли всі попередні тести \mbox{1--15}. 
У\nolinebreak[3] них виконуються такі додаткові обмеження: 
$1\dib{{<}}x\dib{{<}}A\dib{{<}}123$,\hspace{0.5em plus 0.5em}
$1\dib{{<}}y\dib{{<}}B\dib{{<}}123$,\hspace{0.5em plus 0.5em}
$1\dib{{\<}}N\dib{{\<}}10$.\linebreak[2]\hspace{0.5em plus 0.5em}
Цей\nolinebreak[3] блок може принести 50~балів (10\%).

Тести 21--30 оцінюються блоком, лише якщо успішно пройшли всі попередні тести \mbox{1--20}. 
У\nolinebreak[3] них виконуються такі додаткові обмеження: 
$1\dib{{<}}x\dib{{<}}A\dib{{<}}12345$,\hspace{0.5em plus 0.5em}
$1\dib{{<}}y\dib{{<}}B\dib{{<}}12345$,\hspace{0.5em plus 0.5em}
$0\dib{{\<}}N\dib{{\<}}100$.\linebreak[2]\hspace{0.5em plus 0.5em}
Цей\nolinebreak[3] блок може принести 125~балів (25\%).

Тести 31--40 оцінюються блоком, лише якщо успішно пройшли всі попередні тести \mbox{1--30}. 
У\nolinebreak[3] них виконуються такі додаткові обмеження: 
$1\dib{{<}}x\dib{{<}}A\dib{{<}}10^6$,\hspace{0.5em plus 0.5em}
$1\dib{{<}}y\dib{{<}}B\dib{{<}}10^6$,\hspace{0.5em plus 0.5em}
$0\dib{{\<}}N\dib{{\<}}2020$.\linebreak[2]\hspace{0.5em plus 0.5em}
Цей\nolinebreak[3] блок може принести 100~балів (20\%).

Тести 41--50 оцінюються блоком, лише якщо успішно пройшли всі попередні тести \mbox{1--40}.
У\nolinebreak[3] них виконуються такі додаткові обмеження: 
$1\dib{{<}}x\dib{{<}}A\dib{{<}}10^7$,\hspace{0.5em plus 0.5em}
$1\dib{{<}}y\dib{{<}}B\dib{{<}}10^7$,\hspace{0.5em plus 0.5em}
$0\dib{{\<}}N\dib{{\<}}98765$.\linebreak[2]\hspace{0.5em plus 0.5em}
Цей\nolinebreak[3] блок може принести 100~балів (20\%).

В усіх тестах усіх блоків виконується також обмеження 
${1\,{\<}\,r\,{\<}\,\min(A,B)}$. 

\Notes
Зображення наведене лише для кращого розуміння умови, \mbox{Ваша} програма малювати його не~повинна.

Описи блоків наведені лише для пояснення системи оцінювання; \mbox{Ваша} програма не~зобов'язана аналізувати, тест з якого блоку вона обробляє, і кожна \mbox{Ваша} програма-спроба буде оцінюватися заново на всіх тестах від початку й або до кінця, або доки не~виявиться, що якийсь із блоків не~пройдено.

\end{problemAllDefault}

}
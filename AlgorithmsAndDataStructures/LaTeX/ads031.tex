
Учасників полярної експедиції, які зимували на крижині, спіткало велике нещастя: крижина розкололася, і всі вони опинилися на маленькому її уламку. Потрібно було якнайшвидше переправитися через широку тріщину. У їх розпорядженні є двомісний надувний човен. Для кожного полярника відомий час, за який він може переправитися на цьому човні через тріщину. Якщо ж у човні перебувають 2~полярники, час переправи дорівнює часу менш
розторопного з них (тобто того, хто потребує більше часу). За який мінімальний час всі полярники можуть переправитися на велику крижину?

\InputFile
Програма зчитує у першому рядку натуральне число $N$ ($3\leqslant N\leqslant 1000$) --- кількість полярників, а у другому рядку через пропуски --- $N$ натуральних чисел, не~більших 10000, які задають час переправи кожного полярника.

\OutputFile
Програма виводить одне число --- шукану мінімальну тривалість переправи.

\Examples
\begin{example}
\exmp{4
1 6 7 8}{23}
\exmp{4
300 500 800 1000}{2800}
\end{example}

\Note
Тут треба грамотно скомбінувати {\it дві} жадібні стратегії. Перший тест з умови розв'язується однією з них, др{\it у}гий --- іншою з них, решта тестів --- або однією з цих самих двох стратегій, або грамотною комбінацією обох цих жадібних стратегій.

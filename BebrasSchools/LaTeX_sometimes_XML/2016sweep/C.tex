\begin{problemAllDefault}{Дерева}

В Ужляндії пройшов сильний ураган, що звалив дуже багато дерев уздовж дороги. Мер міста занепокоєний цим, він хоче якнайшвидше відремонтувати пошкоджені дороги. Для цього йому потрібно терміново дізнатись сумарну довжину пошкоджених доріг. Відомо лише те, що впало рівно $N$ дерев, також відомі відрізки дороги, на які впало кожне дерево.

\InputFile
В~\mbox{1-му} рядку знаходиться єдине число $N$ --- кількість дерев, $1{\<}N{\<}10^5$.
В~наступних $N$ рядках записано по два числа --- $L_i$ і~$R_i$. Це~означає, що \mbox{$i$-те} дерево пошкодило ділянку дороги від $L_i$ до~$R_i$, $0\dib{{\<}}L_i\dib{{\<}}R_i\dib{{\<}}10^9$.

\OutputFile
Виведіть єдине число --- сумарну довжину пошкодженої частини дороги.

\Examples

\begin{example}
\exmp{3
0 2
1 3
4 5
}{4
}%
\exmp{2
0 10
1 4
}{10
}%
\end{example}

\end{problemAllDefault}


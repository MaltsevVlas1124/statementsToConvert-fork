\begin{problemNoEjudge}{Сніжинка Коха}

Конкретно цю програму пропонується писати, не~здаючи у~систему автоматичної перевірки.

\begin{figure*}[!h]
{\noindent~\hfill
%
\begin{mfpic}[108]{0}{1}{0}{1}
\pointdef{A}(cosd 0,sind 0)
\pointdef{B}(cosd 60,sind 60)
\pointdef{C}(cosd 120,sind 120)
\pointdef{D}(cosd 180,sind 180)
\pointdef{E}(cosd 240,sind 240)
\pointdef{F}(cosd 300,sind 300)
\turtle{(0,0),\A}
\tcaption{$n=0$}
\end{mfpic}
\hfill
\begin{mfpic}[36]{0}{3}{0}{3}
\pointdef{A}(cosd 0,sind 0)
\pointdef{B}(cosd 60,sind 60)
\pointdef{C}(cosd 120,sind 120)
\pointdef{D}(cosd 180,sind 180)
\pointdef{E}(cosd 240,sind 240)
\pointdef{F}(cosd 300,sind 300)
\turtle{(0,0),\A,\B,\F,\A}
\tcaption{$n=1$}
\end{mfpic}
\hfill
\begin{mfpic}[12]{0}{9}{0}{9}
\pointdef{A}(cosd 0,sind 0)
\pointdef{B}(cosd 60,sind 60)
\pointdef{C}(cosd 120,sind 120)
\pointdef{D}(cosd 180,sind 180)
\pointdef{E}(cosd 240,sind 240)
\pointdef{F}(cosd 300,sind 300)
\turtle{(0,0),\A,\B,\F,\A,%
\B,\C,\A,\B,%
\F,\A,\E,\F,%
\A,\B,\F,\A}
\tcaption{$n=2$}
\end{mfpic}
\hfill
\begin{mfpic}[4]{0}{27}{0}{27}
\pointdef{A}(cosd 0,sind 0)
\pointdef{B}(cosd 60,sind 60)
\pointdef{C}(cosd 120,sind 120)
\pointdef{D}(cosd 180,sind 180)
\pointdef{E}(cosd 240,sind 240)
\pointdef{F}(cosd 300,sind 300)
% \turtle{(0,0),\A,\B,\F,\A,%
% \B,\C,\A,\B,%
% \F,\A,\E,\F,%
% \A,\B,\F,\A,%
% %
% \B,\C,\A,\B,%
% \C,\D,\B,\C,%
% \A,\B,\F,\A,%
% \B,\C,\A,\B,%	
% %
% \F,\A,\E,\F,%
% \A,\B,\F,\A,%
% \E,\F,\D,\E,%
% \F,\A,\E,\F,%
% %
% \A,\B,\F,\A,%
% \B,\C,\A,\B,%
% \F,\A,\E,\F,%
% \A,\B,\F,\A}
\turtle{(0,0),\A,\B,\F,\A,%
\B,\C,\A,\B,%
\F,\A,\E,\F,%
\A,\B,\F,\A}

\turtle{(9,0),\B,\C,\A,\B,%
\C,\D,\B,\C,%
\A,\B,\F,\A,%
\B,\C,\A,\B}	

\turtle{(13.5,9*sind 60),\F,\A,\E,\F,%
\A,\B,\F,\A,%
\E,\F,\D,\E,%
\F,\A,\E,\F}

\turtle{(18,0),\A,\B,\F,\A,%
\B,\C,\A,\B,%
\F,\A,\E,\F,%
\A,\B,\F,\A}
\tcaption{$n=3$}
\end{mfpic}
\hfill%
~\linebreak}
\vspace{-0.7\baselineskip}
\caption{Сніжинки Коха порядків 0, 1, 2 та~3.}
\label{fig:koch-ex-0123}
\end{figure*}

За формальне означення самоподібної ламаної <<сніжинка Коха>> можна взяти будь-яке з таких:

А)~Сніжинка Коха нульов\'{о}го порядку є відрізком, а~будь-який \mbox{$n$-й} порядок (де\nolinebreak[3] $n$\nolinebreak[3] --- ціле строго додатне) можна отримати зі сніжинки Коха $(n\,{-}\,1)$-го порядку заміною кожного відрізка на ламану, за такими правилами: відрізок ділиться на три рівні частини, на~середній будують рівносторонній трикутник і замнюють основу трикутника на решту дві сторони.

Б)~Сніжинка Коха нульов\'{о}го порядку є відрізком, а~будь-який \mbox{$n$-й} порядок (де\nolinebreak[3] $n$\nolinebreak[3] --- ціле строго додатне) складається з чотирьох сніжинок Коха \mbox{$(n\,{-}\,1)$-го} порядку втричі менших лінійних розмірів, причому др\'{у}га частина повернута відносно першої на~$60^\circ$\nolinebreak[2] ліворуч, третя відносно др\'{у}гої\nolinebreak[3] --- на~$120^\circ$\nolinebreak[2] праворуч, а~четверта орієнтована так, як перша (тобто повернута відносно третьої на~$60^\circ$\nolinebreak[3] \mbox{ліворуч}).

Напишіть програму, що зображає \mbox{$n$-й} порядок сніжинки Коха.
Рекомендується писати програму мовою Python з використанням <<черепашої графіки>>, тобто виконавця, який вміє виконувати команди 
<<проїхати вперед на таку-то відстань>>, 
<<повернути ліворуч на стільки-то градусів>>, 
<<повернути праворуч на стільки-то градусів>>
(і~ще~деякі, але вони з\'{а}раз не~важливі).

\end{problemNoEjudge}


\begin{problemNoEjudge}{Драконова ламана}

Конкретно цю програму пропонується писати, не~здаючи у~систему автоматичної перевірки.

\newcounter{dragdir}
\newcounter{currx}
\newcounter{curry}
\newcounter{prevx}
\newcounter{prevy}
\def\ddx{\ifcase\value{dragdir} 0 \or 1 \or 0 \or -1 \fi}
\def\ddy{\ifcase\value{dragdir} 1 \or 0 \or -1 \or 0 \fi}


\def\dragR{\ifnum \value{dragdir} < 3 \addtocounter{dragdir}{1} \else \addtocounter{dragdir}{-3} \fi}
\def\dragL{\ifnum \value{dragdir} < 1 \addtocounter{dragdir}{3} \else \addtocounter{dragdir}{-1} \fi}


\def\dragzero{%
\setcounter{prevx}{\value{currx}}
\setcounter{prevy}{\value{curry}}
\pointdef{A}(\arabic{prevx},\arabic{prevy})
\addtocounter{currx}{\ddx}
\addtocounter{curry}{\ddy}
\pointdef{B}(\arabic{currx},\arabic{curry})
\lines{\A,\B}
}

\def\dragone{%
\dragzero
\dragR
\dragzero
}
\def\dragoneR{%
\dragzero
\dragL
\dragzero
}

\def\dragtwo{%
\dragone
\dragR
\dragoneR
}
\def\dragtwoR{%
\dragone
\dragL
\dragoneR
}

\def\dragthree{%
\dragtwo
\dragR
\dragtwoR
}
\def\dragthreeR{%
\dragtwo
\dragL
\dragtwoR
}

\def\dragfour{%
\dragthree
\dragR
\dragthreeR
}
\def\dragfourR{%
\dragthree
\dragL
\dragthreeR
}

\def\dragfive{%
\dragfour
\dragR
\dragfourR
}
\def\dragfiveR{%
\dragfour
\dragL
\dragfourR
}

\def\dragsix{%
\dragfive
\dragR
\dragfiveR
}
\def\dragsixR{%
\dragfive
\dragL
\dragfiveR
}

\def\dragseven{%
\dragsix
\dragR
\dragsixR
}
\def\dragsevenR{%
\dragsix
\dragL
\dragsixR
}

\def\drageight{%
\dragseven
\dragR
\dragsevenR
}
\def\drageightR{%
\dragseven
\dragL
\dragsevenR
}

\def\dragnine{%
\drageight
\dragR
\drageightR
}
\def\dragnineR{%
\drageight
\dragL
\drageightR
}

\def\dragten{%
\dragnine
\dragR
\dragnineR
}


\begin{figure*}[!h]
\begin{center}
\raisebox{28pt}{\begin{mfpic}[20]{0}{12}{0}{12}

\setcounter{currx}{0}
\setcounter{curry}{0}
\setcounter{dragdir}{1}
\dragzero

\setcounter{currx}{3}
\setcounter{curry}{0}
\setcounter{dragdir}{1}
\dragone

\setcounter{currx}{6}
\setcounter{curry}{0}
\setcounter{dragdir}{1}
\dragtwo

\setcounter{currx}{11}
\setcounter{curry}{0}
\setcounter{dragdir}{1}
\dragthree

\setcounter{currx}{18}
\setcounter{curry}{0}
\setcounter{dragdir}{1}
\dragfour

\end{mfpic}}
\hfill
\begin{mfpic}[2]{0}{12}{0}{12}

\setcounter{currx}{0}
\setcounter{curry}{0}
\setcounter{dragdir}{1}
\dragten

\end{mfpic}

\end{center}
\caption{Драконові ламані порядків 0, 1, 2, 3, 4 та, у~дрібнішому масштабі,~10.}
\label{fig:fractal-drag-ex-01234}
\end{figure*}


\begin{figure*}[!h]
\begin{center}
\begin{mfpic}[24]{0}{12}{0}{12}

\def\hsqrtt{0.7071}

% rotated by changing names
\pointdef{LLL}(0,1)
\pointdef{UUU}(1,0)
\pointdef{RRR}(0,-1)
\pointdef{DDD}(-1,0)

\turtle{(0,0),\UUU,\RRR,\DDD,\RRR}
\tlabel[bl](0,0){$A$}
\tlabel[bl](0,-2){$B$}


\tlabel[cc](2,-1){$\Rightarrow$}

\turtle{(5,0),\UUU,\RRR,\DDD,\RRR}
\dashed\turtle{(5,-2),(-\hsqrtt,+\hsqrtt),(+\hsqrtt,+\hsqrtt),(-\hsqrtt,+\hsqrtt),(-\hsqrtt,-\hsqrtt)}
\dashed\arrow\arc[p]{(5,-2),90,180,0.7}
\tlabel[bl](5,0){$A$}
\tlabel[bl](5,-2){$B$}
\tlabel[cr](3.6,-0.5){$A'$}


\tlabel[cc](7.5,-1){$\Rightarrow$}

\turtle{(11,0),\UUU,\RRR,\DDD,\RRR,\DDD,\LLL,\DDD,\RRR}
\tlabel[bl](11,0){$A$}
\tlabel[bl](11,-2){$B$}
\tlabel[br](9,-2){$A'$}

\end{mfpic}
\end{center}
\caption{Процес побудови драконової ламаної \mbox{$n$-го} порядку з ${(n{-}1)}$-го.}
\label{fig:fractal-drag-transform}
\end{figure*}

Самоподібну <<драконову ламану>> можна означити так:

Драконова ламана 
% нульов\'{о}го 
0-го
порядку є відрізком. 
Драконову ламану \mbox{$n$-го} (${n\,{\>}\,1}$) порядку можна отримати з драконової 
ламаної ${(n\,{-}\,1)}$-го порядку так: <<розщепити>> лінію
на~дві, й одну її частину залишити на~місці, а~іншу повернути 
відносно хвоста ламаної ${(n\,{-}\,1)}$-го порядку на~$90^\circ$ проти 
годинникової стрілки.

Напишіть програму, що зображає \mbox{$n$-й} порядок драконової ламаної.
Рекомендується писати програму мовою Python з використанням <<черепашої графіки>>, тобто виконавця, який вміє виконувати команди 
<<проїхати вперед на таку-то відстань>>, 
<<повернути ліворуч на стільки-то градусів>>, 
<<повернути праворуч на стільки-то градусів>>
(і~ще~деякі, але вони з\'{а}раз не~важливі).

\end{problemNoEjudge}


\begin{problemAllDefault}{Спільні дотичні--1}

Як відомо, дотичною до кола є пряма, яка має рівно одну спільну точку з цим колом. Можлива ситуація, коли одна й та сама пряма є дотичною  відразу до двох кіл. Тоді вона називається спільною дотичною. Напишіть програму, яка знаходитиме кількість різних спільних дотичних для заданих двох кіл. При виведенні \mbox{врахуйте} стародавню традицію приписувати числу~7 значення <<багато>>. Тобто, коли кількість спільних дотичних строго більша~6, незалежно від справжньої кількості виводьте~7.

\InputFile
Шість цілих чисел, розділених пропусками (пробілами) $X1$, $Y1$, $R1$, $X2$, $Y2$, $R2$\nolinebreak[3] --- відповідно координати центра і радіуси \mbox{1-го} і \mbox{2-го} кола. Для всіх координат, абсолютна величина (модуль) не~перевищує мільйон. Для обох радіусів, значення у межах від~1 до мільйона (обидві межі включно).

\OutputFile
Програма виводить єдине число\nolinebreak[3] --- шукану кількість, з~ урахуванням згаданої стародавньої традиції.

\Example

\begin{example}
\exmp{20 0 4 50 0 10}{4}%
\end{example}



\end{problemAllDefault}
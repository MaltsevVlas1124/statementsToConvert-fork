\begin{problemAllDefault}{Кількість дільників на проміжку}

Напишіть програму, яка знайде суму кількостей дільників усіх чисел у проміжку від\nolinebreak[2] $A$ до~$B$ (обидві межі включно).

\InputFile У єдиному рядку через пробіл задані два натуральні числа $A$ та~$B$ ($1\,{\<}\,A\,{\<}\,B\,{\<}\,10^{12}$), які 
% являють собою м\'{е}жі 
є межами
проміжку.

\myflfigaw{\hspace*{-0.5em}\begin{exampleSimple}{3.5em}{3em}%
\input K-ex
\end{exampleSimple}\hspace*{-0.25em}}

\OutputFile Виведіть єдине число\nolinebreak[3] --- суму кількостей дільників усіх чисел проміжку.

\Note
Число\nolinebreak[3] 119 має 4\nolinebreak[3] дільники (1,~7, 17,~119);
число\nolinebreak[3] 120 має 16\nolinebreak[3] дільників (1,~2, 3, 4, 5, 6, 8, 10, 12, 15, 20, 24, 30, 40, 60,~120);
число\nolinebreak[3] 121 має 3\nolinebreak[3] дільника (1,\nolinebreak[3] 11,\nolinebreak[3] 121);
число\nolinebreak[3] 122 має 4\nolinebreak[3] дільники (1,~2, 61,~122).
Звідси відповідь \mbox{$4{+}16{+}3{+}4 = 27$}.

\end{problemAllDefault}

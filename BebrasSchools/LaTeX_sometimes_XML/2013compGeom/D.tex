\begin{problemAllDefault}{Яка частина прямої у крузі?}

Є коло (задане радiусом і координатами центра) i пряма (задана координатами двох своїх точок).

Якої довжини вiдрiзок прямої лежить у крузі (всерединi кола)?

\InputFile слід прочитати зі стандартного входу (клавіатури). У~першому рядку задано три числа: спочатку радіус кола~$R$, потім координати його центра $Cx$~$Cy$. У другому та третьому задано по два числа --- $x$- та $y$-координати точок (гарантовано двох різних), через які проходить пряма. Всi числа цiлi, за абсолютним значенням не~перевищують 10000.

\OutputFile Вивести єдине число: якщо пряма і коло мають хоча б одну спільну точку --- довжину вiдрiзка цієї прямої, що лежить у крузі (всерединi кола); якщо не~мають жодної спільної точки --- замість цієї довжини вивести число~\texttt{-1}.

У випадку торкання прямої до кола, спільна точка~є, але відрізка ненульової довжини нема; отже, при торканні слід виводити~\texttt{0}.

Результат при виведенні \emph{не~можна} заокруглювати (а~виводити в експоненційній формі, як-то \texttt{6.0000000000000000E+0000} замість~\texttt{6}, можна).


\Example

\begin{example}
\exmp{
5 0 0
4 1
4 2
}{
6
}\end{example}

\end{problemAllDefault}

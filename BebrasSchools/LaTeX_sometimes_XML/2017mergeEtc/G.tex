\begin{problemAllDefault}{Сума та різниця Мінковського}

\begin{small}

Наведемо два різні формулювання однієї й тієї самої задачі.

\noindent\hrulefill

Сума Мінковського двох фігур на координатній площині $A$, $B$\nolinebreak[3] --- це фігура на координатній площині, якій належать ті й тільки ті точки, координати яких $(x,y)$  можна подати як 
$x\dib{{=}}{x_1{+}x_2}$,
$y\dib{{=}}{y_1{+}y_2}$,
де $(x_1,y_1)$\nolinebreak[3] --- довільна з точок, що належать фігурі~$A$, 
а $(x_2,y_2)$\nolinebreak[3] --- довільна з точок, що належать фігурі~$B$.
Ця\nolinebreak[3] операція можлива для будь-яких фігур $A$,~$B$.

Різниця Мінковського двох фігур на координатній площині $A$, $B$\nolinebreak[3] --- це така фігура~$C$ на координатній площині, що фігура~$A$ є сумою Мінковського фігур~$B$ та~$C$. 
Ця\nolinebreak[3] операція означена лише для деяких фігур $A$,~$B$ (а~для\nolinebreak[2] решти пар $A$,~$B$ підібрати таку~$C$ неможливо).

Напишіть програму, яка читатиме описи фігур $A$, $B$, і знаходитиме суму та різницю Мінковського цих фігур. На вхід гарантовано подаватимуться лише фігури, котрі є опуклими многокутиками, причому одна зі сторін такого многокутика горизонтальна, ліва вершина цієї сторони має координати $(0,\,0)$, а~решта сторін знаходяться у півплощині з додатними~$y$.

Напишіть програму, котра вмітиме знаходити:
\begin{enumerate}
% \begin{shortitems}
\item
суму Мінковського;
\item
різницю Мінковського
% \end{shortitems}
\end{enumerate}

\InputFile
Спочатку \texttt{1} або \texttt{2} на~позначення того, яку задачу потрібно розв'язувати (``1''~--- суму, ``2''~--- різницю), потім описи опуклих многокутників $A$ та $B$, у~такому форматі: спочатку число $N$ (${3{\<}N{\<}1000}$)\nolinebreak[3] --- кількість вершин у многокутнику, а далі $N$ груп по 2 цілих числ\'{а} $x_i$ та~$y_i$\nolinebreak[3] --- координати вершин. Система координат завжди вибирається так, що перша вершина має координати $(0; 0)$, остання має координати $(0;x_n)$, де $x_n{>}0$. Всі вхідні координати\nolinebreak[3] --- цілі ч\'{и}сла, що не~перевищують за модулем (абсолютною величиною)~$10^6$ (деякі з $x_i$ можуть бути від'ємними). 

Всі ч\'{и}сла записано в один рядок і розділено пропусками (пробілами).

\OutputFile
Якщо розв'язується задача~``2'' і різниці для вказаних многокутників не~існує, то вивести на екран єдине число~0. Інакше вивести результат суми чи різниці Мінковського, в тому ж форматі, що для вхідних даних.
Гарантовано, що використовуються лише такі вхідні дані, що якщо відповідь існує, то всі її координати є цілочисельними.


\noindent\hrulefill

На станції Глупов-Товарний використовуються підйомні крани спеціальної конструкції ``Мостовий-Глуповський''. Крюк такого крану підвішений до кількох блоків, що їздять по рейці, розміщеній горизонтально (на певній висоті). Завдяки цьому, крюк можна переміщати в будь-яку точку частини площини, обмежену многокутником спеціального вигляду: верхня сторона многокутника збігається з рейкою крана, обидва внутрішні кути многокутника при цій стороні гострі, решта вершин многокутника розміщені довільним чином, але так, що многокутник виявляється опуклим. Крім того,станція має в розпорядженні пристрій, котрий дозволяє комбінувати дію двох кранів такого типу: простір досяжності крюка скомбінованого механізму точно такий, якби рейку другого крана підвісили (зі збереженням горизонтальності) на крюк першого.

Напишіть програму, котра вмітиме виконувати такі дві дії:
\begin{enumerate}
% \begin{shortitems}
\item
За заданими областями досяжностей двох кранів знаходити область досяжності їхньої комбінації.
\item
За заданими областю досяжності одного крану та потрібною областю досяжності, з'ясовувати, який потрібно взяти другий кран, щоб комбінація першого та другого кранів в точності співпадала з потрібною областю досяжності (або з'ясовувала, що це неможливо).
% \end{shortitems}
\end{enumerate}

\InputFile
Спочатку \texttt{1} або \texttt{2} (на~позначення того, яку задачу потрібно розв'язувати), потім йдуть дві області. Якщо розв'язується задача~``1'', то це області досяжності першого та другого кранів, якщо~``2'', то спочатку потрібна область досяжності, а~потім область досяжності першого крану. Усі області досяжності задаються у такому форматі: спочатку число $N$ (${3{\<}N{\<}1000}$)\nolinebreak[3] --- кількість вершин у многокутнику, а далі $N$ груп по 2 цілих числ\'{а} $x_i$ та~$y_i$\nolinebreak[3] --- координати вершин цієї області в порядку зростання $x$-координати. Система координат завжди вибирається так, що перша вершина має координати (0; 0), вісь~$y$ напрямлена згори донизу. Всі вхідні координати\nolinebreak[3] --- цілі ч\'{и}сла, що не~перевищують за модулем (абсолютною величиною)~$10^6$. 

Всі ч\'{и}сла записано в один рядок і розділено пропусками (пробілами).

\OutputFile
Якщо розв'язується задача~``2'' і підібрати другий кран неможливо, то вивести на екран єдине число~0. Інакше вивести побудовану область досяжності (в тому ж форматі, що для вхідних даних).
Гарантовано, що використовуються лише такі вхідні дані, що якщо відповідь існує, то всі її координати є цілочисельними.

\end{small}




\Examples

\noindent\hspace*{-1mm}\begin{exampleSimple}{24em}{12em}
\exmp{2 5 0 0 10 40 20 50 30 40 40 0 3 0 0 10 10 20 0}{3 0 0 10 40 20 0}%
\exmp{2 3 0 0 10 10 20 0 3 0 0 10 10 40 0}{0}%
\exmp{1 3 0 0 10 10 20 0 3 0 0 10 10 40 0}{4 0 0 20 20 50 10 60 0}%
\end{exampleSimple}

\Note
Пояснимо останній з наведених прикладів вхідних даних та результатів детальніше.
Самі доданки та сума мають такий вигляд (за умови, що вісь~$y$ напрямлена знизу догори):

\vspace{-0.875\baselineskip}

~
\begin{mfpic}[24.0]{0}{2}{0}{2}
\polygon{(0,0),(1,1),(2,0)}
\tlabel[tr](0,0){$A_1$}
\tlabel[bl](1,1){$A_2$}
\tlabel[tl](2,0){$A_3$}
\end{mfpic}
%
\hfill
\begin{Huge}$+$\end{Huge}
\hfill
%
\begin{mfpic}[24.0]{0}{4}{0}{2}
\polygon{(0,0),(1,1),(4,0)}
\tlabel[tr](0,0){$B_1$}
\tlabel[bl](1,1){$B_2$}
\tlabel[tl](4,0){$B_3$}
\end{mfpic}
%
\hfill
\begin{Huge}$=$\end{Huge}
\hfill
%
{\begin{mfpic}[24.0]{0}{6}{0}{2}
\polygon{(0,0),(2,2),(5,1),(6,0)}
\tlabel[tr](0,0){$C_1$}
\tlabel[bl](2,2){$C_2$}
\tlabel[bl](5,1){$C_3$}
\tlabel[tl](6,0){$C_4$}
\end{mfpic}}
\hfill
~
\\

Цей результат може бути утворений, наприклад, внаслідок <<ковзань>> др\'{у}гого многокутика уздовж сторін першого.

~
\hfill
%
\begin{mfpic}[20]{0}{6}{-2}{0}
\pen{3pt}\polygon{(0,0),(2,2),(5,1),(6,0)}\pen{0.5pt}
\polygon{(0,0),(1,1),(4,0)}
\polygon{(0.3,0.3),(1.3,1.3),(4.3,0.3)}
\polygon{(0.6,0.6),(1.6,1.6),(4.6,0.6)}
% \polygon{(0.8,0.8),(1.8,1.8),(4.8,0.8)}
\polygon{(1.0,1.0),(2.0,2.0),(5.0,1.0)}
\dotted\lines{(4,0),(5,1)}
\tcaption{(a)}
\end{mfpic}
%
\hfill
%
\begin{mfpic}[20]{0}{6}{-2}{0}
\pen{3pt}\polygon{(0,0),(2,2),(5,1),(6,0)}\pen{0.5pt}
\polygon{(2,0),(3,1),(6,0)}
\polygon{(1.7,0.3),(2.7,1.3),(5.7,0.3)}
\polygon{(1.4,0.6),(2.4,1.6),(5.4,0.6)}
% \polygon{(1.2,0.8),(2.2,1.8),(5.2,0.8)}
\polygon{(1.0,1.0),(2.0,2.0),(5.0,1.0)}
\dotted\lines{(2,0),(1,1)}
\tcaption{(b)}
\end{mfpic}
\hfill
%
\begin{mfpic}[20]{0}{6}{-2}{0}
\pen{3pt}\polygon{(0,0),(2,2),(5,1),(6,0)}\pen{0.5pt}
\polygon{(0,0),(1,1),(4,0)}
\polygon{(1,0),(2,1),(5,0)}
% \polygon{(1.5,0),(2.5,1),(5.5,0)}
\polygon{(2,0),(3,1),(6,0)}
\dotted\lines{(1,1),(3,1)}
\tcaption{(c)}
\end{mfpic}
%
\hfill
~
\\


\end{problemAllDefault}
\begin{problemAllDefault}{Бінпошук у масиві--1}

Дано два масиви. Гарантовано, що у першому з них всі елементи різні. Гарантовано, що др\'{у}гий містить лише елементи, які є в першому (але, можливо, не всі). Відсортуйте перший з них, а потім для кожного з елементів др\'{у}гого масиву знайдіть, під яким номером (нумерація починається з~1) він знаходиться у відсортованому (за~зростанням) першому масиві.

\InputFile
\mbox{1-й}\nolinebreak[3] рядок містить єдине число $N$ ($1{\<}N{\<}123456$) --- кількість елементів \mbox{1-го}\nolinebreak[3] масиву. \mbox{2-й}\nolinebreak[3] рядок містить $N$ розділених пропусками (пробілами) гарантовано різних чисел --- елементи цього масиву. Не~менш ніж \mbox{50\%}\nolinebreak[3] балів припадає на тести, де цей масив задано вже відсортованим (за~зростанням), але у~решті тестів це не~так.

\mbox{3-й}\nolinebreak[3] рядок містить єдине число $M$ ($1{\<}M{\<}N$)\nolinebreak[3] --- кількість елементів \mbox{2-го}\nolinebreak[3] масиву. \mbox{4-й}\nolinebreak[3] рядок містить $N$ розділених пропусками (пробілами) гарантовано різних чисел\nolinebreak[3] --- елементи цього масиву. Кожен з елементів \mbox{2-го}\nolinebreak[3] масиву гарантовано зустрічається також і~в~\mbox{1-му}.

Значення елементів обох масивів є цілими числами, що не~перевищують за модулем $10^9$ (мільярд).

\OutputFile
Виведіть у один рядок через пропуски (пробіли) рівно $M$ чисел: яким за номером у відсортованому вигляді \mbox{1-го}\nolinebreak[3] масиву є \mbox{1-й}\nolinebreak[3] елемент \mbox{2-го}\nolinebreak[3] масиву, яким за номером у відсортованому вигляді \mbox{1-го}\nolinebreak[3] масиву є \mbox{2-й}\nolinebreak[3] елемент \mbox{2-го}\nolinebreak[3] масиву,\nolinebreak[2] \dots, яким за номером у відсортованому вигляді \mbox{1-го}\nolinebreak[3] масиву є \mbox{$M$-й}\nolinebreak[3] елемент \mbox{2-го}\nolinebreak[3] масиву. \mbox{2-й}\nolinebreak[3] масив сортувати не~слід, це~повинні бути відповіді для того \mbox{2-го}\nolinebreak[3] масиву, який задано у вхідних даних.

\Example

\begin{example}
\exmp{5
7 17 42 23 13
3
17 42 23
}{3 5 4 
}%
\end{example}

\end{problemAllDefault}


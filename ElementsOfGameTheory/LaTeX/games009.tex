Є три к{\it у}пки, кожна з яких містить деяку кількість паличок.
Двоє грають у таку гру.
Кожен з гравців на кожному своєму ході може забрати з будь-якої однієї купки будь-яку кількість паличок, від~1 до зразу всіх паличок цієї купки. Палички можна лише забирати (ні~додавати, ні~перекладувати з~ купки в~купку не~можна).
Ніяких інших варіантів ходу нема. 
Коли купка стає порожньою (кількість паличок=0), гра просто продовжується для решти купок.
Ходять гравці по черзі, пропускати хід не~можна.
Виграє той, хто забирає останню паличку (можливо, разом із ще деякими) з останньої купки.
(Інакше кажучи, виграє той, після чийого ходу не~лишається жодної палички в жодній купці.)

Напишіть програму, яка визначатиме, хто виграє при правильній грі обох гравців. 
Іншими словами, хто може забезпечити собі виграш, хоч~би~як не~грав інший.

\InputFile
Єдиний рядок містить рівно три цілих числ{\it а}, кожне в~діапазоні від~1 до~40~--- початкові кількості паличок у~кожній з~купок. Серед них можуть бути як однакові, так і різні.


\OutputFile
Єдине ціле число, або \texttt{1} (якщо перший гравець може забезпечити собі виграш), або \texttt{2} (якщо др{\it у}гий).

\Examples
\begin{example}
\exmp{2 3 4}{1}%
\exmp{2 5 5}{1}%
\exmp{1 2 3}{2}%
\end{example}

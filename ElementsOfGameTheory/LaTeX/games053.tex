% \begin{problemAllDefault}{Кілька фішок на мінному полі}
% \label{problem:minefield-several}

Правила гри переважно відповідають задачі\nolinebreak[3] 
% \ref{problem:minefield-1} 
«Фішка на мінному полі—1» (включно з \textsl{«фішка може рухатися або праворуч, або вниз, на будь-яку кількість клітинок у~межах \mbox{п\'{о}ля}, не~стаючи на міни й не~перестрибуючи~їх»} та «нормальною умовою завершення», тобто \textsl{«хто не~може ходити\nolinebreak[3] — програ\'{є}»}), але спочатку маємо $k$ фішок, які стоять на різних неза\-мі\-но\-ва\-них клітинках. Кожен гравець на кожному своєму ході може рухати будь-яку фішку, але лише одну. Перестрибувати іншу фішку (чи кілька інших фішок) можна. Якщо різні фішки потрапляють в одну клітинку, вони нега\-йно щезають.

Напишіть програму, яка визначить, хто в\'{и}грає при правильній грі обох гравців, а\nolinebreak[3] якщо в\'{и}грає \mbox{1-й} гравець, то також знайде сукупність його «виграшних ходів» (після яких він все~ще не~втрачає свій виграш при правильній грі обох гравців).

\InputFile
Перший рядок містить два цілі числ\'{а} $N$ та~$M$, розділені одним пробілом\nolinebreak[3] — спочатку кількість рядків, потім кількість стовпчиків. Обидва ці значення у межах від~1 до~123.
Далі йдуть $N$ рядків, що задають мінне поле. Кожен з них містить рівно по~$M$ символів \texttt{.} (позначає вільну клітинку) та/або \texttt{*} (позначає заміновану клітинку). Ці~символи йдуть без роздільників, і кожен з цих $N$ рядків містить лише ці символи та переведення рядка наприкінці.

Далі окремим рядком записане ціле число~$k$ ($1\dib{{\<}}k\dib{{\<}}12$)\nolinebreak[3] — початкова кількість фішок. Далі йдуть ще $k$ рядків, кожен з яких містить по два цілі числ\'{а}, розділені одним пробілом\nolinebreak[3] — спочатку номер рядка, потім номер стовпчика початкового розміщення чергової фішки, причому рядки нумеруються від~1 до~$N$ згори донизу, стовпчики\nolinebreak[3] — від~1 до~$M$ зліва направо. Гарантовано, що початкові розміщення вісх фішок різні, й кожна фішка розміщена в неза\-мі\-но\-ва\-ній клітинці.

\OutputFile
Перший рядок має містити єдине ціле число, або \texttt{1} (якщо перший гравець може забезпечити собі виграш), або \texttt{2} (якщо др\'{у}гий). Якщо відповідь з першого рядка~\texttt{2}, то на цьому виведення слід припинити. А~якщо відповідь з першого рядка~\texttt{1}, то далі треба вивести також перелік всіх можливих перших ходів першого гравця, після яких др\'{у}гий (при правильній грі першого) вже ні\'{я}к не~зможе виграти. Цей перелік виводити в такому форматі: в~один рядок через одинарні пробіли номер фішки, напрям (єдина буква ``\texttt{D}''/``\texttt{R}'' без лапок) та кількість клітинок, на які слід перемістити фішку. 
Порядок виведення «виграшних ходів» може бути довільним, але вони мусять бути згадані всі, кожен по одному разу.
Номер фішки слід задавати числом від\nolinebreak[3] 1 до~$k$, в~порядку, як вони задані у вхідних даних.

\Examples

\noindent
\begin{exampleSimpleExtraNarrow}{3em}{4em}
\exmp{2 4
....
.**.
2
1 1
2 4}{1
1 D 1
1 R 2}
\exmp{1 1
.
1
1 1}{2}%
\end{exampleSimpleExtraNarrow}
\begin{exampleSimpleExtraNarrow}{5em}{4em}
\exmp{5 7
....*..
.*.....
.......
.....*.
..*....
4
1 1
2 5
3 3
4 2}{1
1 D 3
2 R 2
3 R 1}%
\end{exampleSimpleExtraNarrow}


% \end{problemAllDefault}
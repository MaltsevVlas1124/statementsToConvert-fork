{\it Єдина відмінність цієї задачі від попередньої --- той, кому нема куди ходити, виграє, а~не~програє. Це~єдина з усіх задач серії «Фішка на мінному полі», де вжите таке (протилежне стандартному) правило визначення виграшу.}

На прямокутному полi $N{\times}M$ клiтинок у лiвому верхньому кутку стоїть фішка. Ця~кутова клітинка гарантовано вільна (не~замінована); абсолютно кожна інша клітинка п{\it о}ля може бути хоч вільною, хоч замінованою. 
Гра полягає в тому, що два гравцi поперемiнно рухають згадану фішку на якусь кiлькiсть клiтинок праворуч або донизу (кожен гравець сам вирiшує, в якому з цих двох напрямків i на скільки клітинок рухати; не~можна ні~лишати фішку на мiсцi, ні~ставати нею в заміновану клітинку, ні~перестрибувати нею через заміновану клітинку).
\underline{\it Виграє} той, хто не~може нiкуди походити (і~знизу, і~праворуч або край п{\it о}ля, або міна). Відповідно, його суперник \underline{\it програє}.

Напишіть програму, яка визначатиме, хто виграє при правильній грі обох гравців. 
Іншими словами, хто може забезпечити собі виграш, хоч би як не грав інший.

\InputFile
Перший рядок містить два цілі числ{\it а} $N$ та~$M$, розділені одним пропуском (пробілом)~--- спочатку кількість рядків, потім кількість стовпчиків. Обидва ці значення у межах від~1 до~12.

Далі йдуть $N$ рядків, що задають мінне поле. Кожен з них містить рівно по~$M$ символів \texttt{.} (позначає вільну клітинку) та/або \texttt{*} (позначає заміновану клітинку). Ці~символи йдуть без роздільників, і кожен з цих $N$ рядків містить лише ці символи та переведення рядка наприкінці.

\OutputFile
Єдине ціле число, або \texttt{1} (якщо перший гравець може забезпечити собі виграш), або \texttt{2} (якщо др\it{у}гий).

\Examples
\begin{example}
\exmp{2 4
....
.**.
}
{1
}
\exmp{1 1
.
}
{1
}
\end{example}

\Note
У першому прикладі, єдиний спосіб, яким перший гравець може виграти~--- піти три клітинки праворуч, після чого др{\it у}гий буде змушений піти на одну клітинку вниз (це~буде єдиний можливий його хід), після чого першому не~буде куди йти, і він в{\it и}грає. 
Таким чином, зміна правила визначення переможця на протилежне не~гарантує зміни переможця на протилежного: гравці знають змінені правила гри, тож {\it змінюють свої ходи} згідно зі зміненими правилами. Водночас, др{\it у}гий приклад показує, що зміна правила визначення переможця на протилежне в~деяких випадках може змінити переможця на протилежного.

Тому, слово «навпаки» вжите в назві задачі для того, щоб підкреслити: дуже часто геть не~ясно, що~взагалі таке «навпаки». Завершити ж усе це хочемо відомою «цитатою зі шкільного учнівського твору»: «на березі річки доярка доїла корову, а у воді відображалось усе навпаки».